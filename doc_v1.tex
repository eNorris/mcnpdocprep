\documentclass[10pt,a4paper]{book}
\usepackage[utf8]{inputenc}
\usepackage{amsmath}
\usepackage{amsfonts}
\usepackage{amssymb}
\usepackage{graphicx}
\author{Edward T. Norris}
\title{An Introduction to MCNP}
\begin{document}

\maketitle

\tableofcontents

\chapter{Introduction}

Why did I write this book? The manual sucks.

\section{Forewarning}
First, reader be forewarned, the contents herein pertain to MCNP6. Since MCNP6 was based directly on MCNP5 (and hence earlier versions), many of the same concepts can be drag-and-dropped verbatim. However, there are some differences between the two. MCNP6 was a merger between the standard at the time, MCNP5 and MCNPX.

<Purpose of 5/X>

Every version of MCNP comes with a comprehensive manual with the code itself.

\section{Obtaining MCNP6}

Go to the RSICC.

\section{Outline}

The structure of this book is built to suit an undergraduate understanding of MCNP6. The emphasis herein is not on the theory and mathematics, though they will be mentioned when they are applicable, but rather on the mechanical operation of MCNP6 as a computational tool to perform analysis. The following chapter will provide a quick summary of the entire book, mapping out the structure of an MCNP6 file as well as the core components needed. Subsequent chapters will focus on a single element of the input file at a time.


\section{Formatting}
This is my book. It will occasionally have comments about giraffes. Because they're fucking awesome.

\section{Logistics}

Uh, yeah, there might be some.

\chapter{A Crash Course}

I suspect that some readers require more of an immediate start as opposed to an in depth documentary on the inricacies of each and every option MCNP6 has to offer. They are more interesting in running a simple problem or modifying an existing input file to perform some additional calculation.

This chapter will do two things; first it will go over the overall structure of a MCNP6 input, second, it will provide a single-chapter crash course on running an input file for those that simple need to run a file.

\chapter{Surfaces}
They are smooth. Usually.
Use pen and paper!

\section*{Problems}

\begin{enumerate}
\item squid.

\item dog.

\item Create a surface.

\end{enumerate}

\chapter{Simple Cells}
Math for the win.

\chapter{Repeated Structures}
Yeah, they repeat.

\section{Like X But Y}
This is like that, but... well, not.

\section{Lattices}
They suck a lot. Yeah, a whole lot.

\section{Universes}
They go inside themselves.

\chapter{Materials}
Like copper, iron, and cotton candy.

\section{Cross Section Libraries}
They have data. Making your own custom libraries.

\chapter{Source Definitions}
This is going to be a really long chapter

\section{Point Sources}
They are simple

\section{Energy Distributions}
There are quite a few types of them

\section{Beam and Cone Sources}
Pencil beams, and cones for ice cream.

\section{Volume Sources}
Lots of them!

\chapter{Tallies}
These are useful.

\section{Surface Tallies}
Across the surface, ho!

\section{Volume Tallies}
Through the volume, ho!

\section{Point Detectors}
They are magical and perfect.

\section{Mesh Tallies}
They're all meshy.

\section{Modeling real detectors}
It's hard. I quit.

\chapter{Visualizing}
If you're blind, just skip this chapter.

\section{MCNP Plotter}
It sucks.

\section{VisEd}
It's not quite as bad as the plotter, but it still sucks. A lot.

\chapter{Postprocessign Data}
Like Matlab?

\chapter{Theory}
You're reading the wrong book. Sorry buddy.

\end{document}














